\documentclass[12pt, a4paper]{report}
% Begin imports
\usepackage{geometry}
\geometry{margin=2cm}
\usepackage{amssymb}
\usepackage{amsmath}
\usepackage[italian]{babel}
\usepackage{amsthm}
\usepackage[most]{tcolorbox}
\usepackage{xcolor}
\usepackage[colorlinks=true, linkcolor=blue]{hyperref}
% End imports

% Begin definitions
    \newtcbtheorem[auto counter,number within=section]{definitionbox}{Definizione}{
        colback=blue!5!white,
        colframe=blue!75!black,
        fonttitle=\bfseries,
        boxsep=5pt,
        arc=5pt,
        boxrule=1pt,
        title={Definizione}
    }{def}

    \tcbset{
        example/.style={
            colback=gray!5!white, 
            colframe=gray!75!black,
            boxsep=5pt,
            arc=5pt,
            boxrule=1pt,
            title={\textbf{Esempio}}
        }
    }
    
    \newenvironment{example}[1][]{\begin{tcolorbox}[example, #1]}{\end{tcolorbox}}
% End definitions

% Begin title page
\title{Note TAC}
\author{}
\date{}
% End title page

\begin{document}
\maketitle
\tableofcontents
\chapter{Preliminari}
\section{Alfabeti, stringhe, linguaggi}
    \begin{definitionbox}{Alfabeto}{}
        Un \textit{alfabeto} ($\Sigma$) è un insieme finito \textbf{non vuoto} di oggetti detti \textit{simboli} ($\sigma$)
    \end{definitionbox}
    \begin{definitionbox}{Stringa}{}
        Un stringa $u$ su un alfabeto $\Sigma$ è una funzione totale da $\left[1,n\right]$ in $\Sigma$ per qualche $n\in\mathbb{N}$; $n$ si dice lunghezza di $u$ e si indica con $\left|u\right|$.\\
        \textbf{Approfondimento}:
        \begin{itemize}
            \item Funzione totale: funzione che per ogni input c'è sempre un output;
            \item Funzione stringa: $i\to\sigma$ con $i\in\mathbb{N}$ posizione nella stringa. 
        \end{itemize} 
    \end{definitionbox}
    L'unica stringa $u$ con $\left|u\right|=0$ si chiama \underline{stringa} vuota e si indica con $\varepsilon$.\\
    L'insieme di tutte le possibili stringhe su $\Sigma$ si indica con $\Sigma^*$ e l'insieme di tutte le possibili stringhe non vuote su $\Sigma$ si indica con $\Sigma^+$. Notiamo che $\Sigma^*$ è sempre un insieme infinito perchè non si ha un limite di lunghezza delle stringhe.
    \begin{definitionbox}{Linguaggio}{}
        Un linguaggio $L$ è un insieme di stringhe selezionate da $\Sigma^*$ (ovvero è sottoinsieme di $\Sigma^*$).
    \end{definitionbox}
    L'insieme vuoto $\emptyset$  e l'insieme costituito solo dalla stringa vuota $\{\varepsilon\}$ sono linguaggi su \textbf{qualunque} alfabeto.
    \newpage
    \subsection{Operazioni su stringhe}
        Siano $u$ e $v$ stringhe di lunghezza $n$ ed $m$ rispettivamente
        \begin{definitionbox}{Concatenazione}{}
            $u\cdot v$ è la stringa di lunghezza $n+m$ definita da:
            $$
                (u\cdot v)(k)=\begin{cases}
                    u(k) & \text{se } 1\leq k\leq n\\
                    v(k-n) & \text{se } n<k\leq n+m\\
                \end{cases}
            $$
            E' un'operazione associativa e ha come identità la stringa vuota ($u\cdot\varepsilon=u$), si tratta quindi di un monoide.
        \end{definitionbox}
        \begin{definitionbox}{Ripetizione}{}
            $u^n$ (stringa $u$ ripetuta $n$ volte) per $n\geq 0$ è definita induttivamente da $u^0=\varepsilon,u^{n+1}=u\cdot u^n$
        \end{definitionbox}
    \subsection{Operazioni su linguaggi}
        Siano $L$ e $L'$ linguaggi
        \begin{definitionbox}{Concatenazione}{}
            $L\cdot L'=\{u\cdot v\mid u\in L,v\in L'\}$, ovvero: la concatenazione è l'insieme di tutte le possibili combinazioni di stringhe.\\
            Questa operazione è associativa e ha come identità l'insieme $\{\varepsilon\}$ mentre l'insieme vuoto costituisce uno zero dell'operazione ($L\cdot\emptyset= \emptyset\cdot L=\emptyset$)
        \end{definitionbox} 
        \begin{example}{}
            \begin{itemize}
                \item $L=\{a,ab\}$
                \item $L'=\{b,ba\}$
            \end{itemize}
            $LL'=\{ab,aba,abb,abba\}$
        \end{example} 
        \begin{definitionbox}{Ripetizione}{}
            $L^n$ per $n\geq 0$  è definito induttivamente da $L^0=\{\varepsilon\},L^{n+1}=L\cdot L^n$.\\
            La chiusura di Kleene $L^*$ è definita da $L^*=\bigcup_{n\geq 0}L^n$, e la chiusura positiva $L^+$  da $L^+=\bigcup_{n>0}L^n$.
        \end{definitionbox}  
\end{document}