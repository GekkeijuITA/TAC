\newcommand{\preamble}{
    \usepackage[italian]{babel}
    \usepackage{geometry}
    \usepackage{amsmath}
    \usepackage{amssymb}
    \usepackage{graphicx}
    \usepackage{ulem}
    \usepackage{amsthm}
    \usepackage{listings}
    \usepackage{xparse}
    \usepackage{expl3}
    \usepackage{tikz}
    \usepackage[colorlinks=true, linkcolor=blue]{hyperref}
    \usetikzlibrary{calc}
    \let\olditemize\itemize
    \renewcommand\itemize{\olditemize\setlength\itemsep{0em}}
    \geometry{a4paper, left=1cm, right=1cm, top=1cm, bottom=2cm}
    \usepackage[most]{tcolorbox}
    \usepackage{xcolor}

    \newtcbtheorem[auto counter,number within=section]{definitionbox}{Definizione}{
        colback=blue!5!white,
        colframe=blue!75!black,
        fonttitle=\bfseries,
        boxsep=5pt,
        arc=5pt,
        boxrule=1pt,
        title={Definizione}
    }{def}

    \newtcbtheorem[auto counter,number within=section]{theorembox}{Teorema}{
        colback=red!5!white,
        colframe=red!75!black,
        fonttitle=\bfseries,
        boxsep=5pt,
        arc=5pt,
        boxrule=1pt,
        title={Teorema}
    }{thm}
    
    \tcbset{
        demonstration/.style={
            colback=magenta!5!white, 
            colframe=magenta!75!black,
            boxsep=5pt,
            arc=5pt,
            boxrule=1pt,
            title={\textbf{Dimostrazione}}
        }
    }
    
    \newenvironment{demonstration}[1][]{
        \begin{tcolorbox}[demonstration, ##1]
    }{
        \end{tcolorbox}
    }

    \tcbset{
        exercise/.style={
            colback=orange!5!white, 
            colframe=orange!75!black,
            boxsep=5pt,
            arc=5pt,
            boxrule=1pt,
            title={\textbf{Esercizio}}
        }
    }
    
    \newenvironment{exercise}[1][]{
        \begin{tcolorbox}[exercise, ##1]
    }{
        \end{tcolorbox}
    }

    \tcbset{
        example/.style={
            colback=gray!5!white, 
            colframe=gray!75!black,
            boxsep=5pt,
            arc=5pt,
            boxrule=1pt,
            title={\textbf{Esempio}}
        }
    }
    
    \newenvironment{example}[1][]{
        \begin{tcolorbox}[example, ##1]
    }{
        \end{tcolorbox}
    }

        \tcbset{
        algorithm/.style={
            colback=yellow!5!white, 
            colframe=yellow!75!black,
            boxsep=5pt,
            arc=5pt,
            boxrule=1pt,
            title={\textbf{Algoritmo}}
        }
    }
    
    \newenvironment{algorithm}[1][]{
        \begin{tcolorbox}[algorithm, ##1]
    }{
        \end{tcolorbox}
    }

    \definecolor{codegreen}{rgb}{0,0.6,0}
    \definecolor{codegray}{rgb}{0.5,0.5,0.5}
    \definecolor{codepurple}{rgb}{0.58,0,0.82}
    \definecolor{backcolour}{rgb}{0.95,0.95,0.92}

    \lstdefinestyle{mystyle}{  
        commentstyle=\color{codegreen},
        keywordstyle=\color{magenta},
        numberstyle=\tiny\color{codegray},
        stringstyle=\color{codepurple},
        basicstyle=\ttfamily\footnotesize,
        breakatwhitespace=false,         
        breaklines=true,                 
        captionpos=b,                    
        keepspaces=true,                   
        numbersep=5pt,                  
        showspaces=false,                
        showstringspaces=false,
        showtabs=false,                  
        tabsize=2
    }

    \lstset{style=mystyle}
}

\newcommand{\equivalenceclass}[1]{
    \left[#1\right]_\sim
}

\newcommand{\quotientof}[1]{
    #1/_\sim
}