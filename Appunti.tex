\documentclass[12pt, a4paper]{report}
\usepackage[utf8]{inputenc}
\newcommand\preamble{
    \usepackage[italian]{babel}
    \usepackage{geometry}
    \usepackage{amsmath}
    \usepackage{amssymb}
    \usepackage{graphicx}
    \usepackage{ulem}
    \usepackage{amsthm}
    \theoremstyle{definition}
    \newtheorem{definition}{Def}[section]
    \usepackage{listings}
    \usepackage{xparse}
    \usepackage{expl3}
    \usepackage{tikz}
    \usepackage[colorlinks=true, linkcolor=blue]{hyperref}
    \usetikzlibrary{calc}
    \let\olditemize\itemize
    \renewcommand\itemize{\olditemize\setlength\itemsep{0em}}
    \geometry{a4paper, left=1.2cm, right=1cm, top=1cm, bottom=2cm}
}
\newcommand{\tikzmark}[1]{\tikz[baseline,remember picture] \coordinate (#1) {};}
\newcommand{\customfbox}[1]{
    \begin{center}
        \noindent\fbox{\parbox{\dimexpr\linewidth-2\fboxsep-2\fboxrule\relax}{\centering #1}}
    \end{center}
    }
\newcommand{\imagePath}{Images/logoUni.png}

\newcommand{\customTitlePage}[5]{
    \newcommand{\courseTitle}{#1}
    \newcommand{\authorName}{#2}
    \newcommand{\academicYear}{#3}
    \newcommand{\universityName}{#4}
    
    \begin{titlepage}
        \centering
        \includegraphics[width=0.5\textwidth]{\imagePath}\par\vspace{1cm}
        {\scshape\LARGE \universityName \par}
        \vspace{1.5cm}
        {\huge\bfseries \courseTitle \par}
        \vspace{2cm}
        {\Large\itshape \authorName \par}
        \vfill
        \academicYear
    \end{titlepage}
}
\preamble

\begin{document}
    \customTitlePage{Teoria degli Automi e Calcolabilità}{Lorenzo Vaccarecci}{Anno Accademico 2024/2025}{Università degli Studi di Genova}
    \newpage
    \tableofcontents
    \chapter{Preliminari}
        \section{Alfabeti, stringhe, linguaggi}
            \begin{definition}[Alfabeto]
                Un alfabeto è un insieme finito non vuoto di oggetti detti simboli.
            \end{definition}
            \begin{definition}[Stringa]
                Una stringa $u$ su un alfabeto $\Sigma$ è una funzione totale da $\left[1,n\right]$ in $\Sigma$ per un qualche $n \in \mathbb{N}$. $n$ si dice lunghezza di $u$ e si indica con $|u|$. Useremo $\sigma$ per indicare generici simboli e $u,v,w$ per indicare generiche stringhe. La stringa vuota ($|u|=0$) si indica con $\Lambda$ o $\varepsilon$. $\Sigma^{*}$ è l'insieme di tutte le stringhe su $\Sigma$ ed è sempre un insieme infinito.
            \end{definition}
            \begin{equation*}
                \begin{split}
                    &\Sigma = \left\{a,b\right\} \rightarrow  \text{insieme di simboli}\\
                    &\Sigma^{*} = \left\{\varepsilon,a,b,aa,ab,\ldots\right\} \rightarrow \text{insieme delle possibili combinazioni di} \Sigma\\
                \end{split}
            \end{equation*}
            \begin{definition}[Linguaggio]
                Un linguaggio su un alfabeto $\Sigma$ è un insieme di stringhe su $\Sigma$, ossia un sottoinsieme di $\Sigma^{*}$. Useremo $L$ per indicare generici linguaggi. L'insieme vuoto $\emptyset$ e l'insieme costituito solo dalla stringa vuota $\left\{\varepsilon\right\}$ sono linguaggi su qualunque alfabeto.
            \end{definition}
            In genere, scriviamo le stringhe utilizzando la rappresentazione per \textbf{giustapposizione}, cioè  semplicemente scrivendo i simboli uno dopo l'altro da sinistra a destra. Questa rappresentazione è arbitraria e può risultare  ambigua, mentre la definizione 1.1.2 è rigorosa e indipendente dalla rappresentazione.
            \begin{definition}[Concatenazione]
                \begin{equation*}
                    \begin{split}
                        &u:\left[1..n\right]\rightarrow \Sigma \quad v:\left[1..m\right]\rightarrow \Sigma\\
                        &u\cdot v:\left[1..n+m\right] \rightarrow \Sigma \\
                        &(u\cdot v)(i) = \begin{cases}
                            u(i) & \text{se } 1\leq i \leq n\\
                            v(i-n) & \text{se } n < i \leq n+m
                        \end{cases}
                    \end{split}
                \end{equation*}
                \textbf{Definizione induttiva}: $u^{0}=\epsilon,u^{n+1}=u\cdot u^{n}$
            \end{definition}
            \begin{definition}[Operazioni su linguaggi]
                Se $L$ e $L'$ sono linguaaggi, $L\cdot L'=\{u\cdot v | u\in L,v\in L'\}$. Scriveremo anche semplicemente $LL'$. Inoltre $L^{n}$, per $n\geq 0$, è definito induttivamente da $L^{0}=\left\{\epsilon\right\},L^{n+1}=L\cdot L^{n}$. Infine, la \textbf{chiusura di Kleene} $L^{*}$ di un linguaggio è definita da $L^{*}=\bigcup_{n\geq 0}L^{n}$, e la \textbf{chiusura positiva} $L^{+}$ da $L^{+}=\bigcup_{n>0}L^{n}$.\\
                Questa operazione ha come identità l'insieme $\left\{\epsilon\right\}$, mentre l''insieme vuoto costituisce uno zero dell'operazione: $L\cdot\emptyset=\emptyset\cdot L=\emptyset$
            \end{definition}
\end{document}